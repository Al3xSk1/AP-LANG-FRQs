\documentclass[10pt]{article}
\usepackage{indentfirst}

\title{AP LANG FRQs}
\author{Alexander Kovalski}
\date{\today}
\begin{document}

\maketitle

\begin{abstract}

Alexander Kovalski's collection AP English Language and Composition Past FRQ answers from 2023 to 1999 Work in Progress.

\end{abstract}

\tableofcontents

\section{2023-2}

\subsection{Question 1}

There is immense value in vertical farming because vertical farming provides extremely high crop yields and is a solution to insufficient farmland.

More availability is better than less availability. Vertical farming provides a wide range of crops access to a controlled environment which allows the crops to regularly retain extremely high yields of crops. Authors of a book on vertical farming, Kozai, Toyoki, and Genhua Niu, showcase their findings that “open field” farming has “low yield” while “vertical farming” has “extremely high yields” (Source C). As a result vertical farming provides significant crop yield per square foot. Take for example, how AppHarvest, a vertical farming company, has “45-foot tall tomato vines” and “more than 10 million heads of lettuce a year” without “a teaspoon of soil.” (Source A). Even renowned cook, Martha Stewart supports vertical farming and AppHarvest. The vertical farming industry is providing Americans with extremely high crop yields, ultimately showcasing how more availability of crops are better for society.

Inefficiencies are met with solutions to solve said inefficiencies. There is a significant lack of farmland in the world. The usable farmland totals the size of South America and provides enough food for the 2010 world population of 6.8 billion people according to Despommier, an expert in vertical farming. However he also claims that we will need an area the size of Brazil to feed another 2.7 billion people, the expected amount by 2050 (Source F). Farmland the size of Brazil cannot just come out of thin air, thus society needed vertical farming. Since vertical farming “allows growers to grow regional or seasonal crops indoors year-round” (Source B) it will provide a significant amount of excess food which cannot be provided by traditional farming methods. All of this highlights the need for vertical farming to solve the Inefficiencies of crops provided by traditional farming.

\subsection{Question 2}

In a speech delivered to the graduating class of 2016 at University of Virginia, Charlottesville, Poet Dove defines her wishes for the graduating class and exemplifies the work of famous people in order to highlight that although the world is “scary”, the graduating class should understand that they need to make the best of their lives so the they can embrace and conquer the unknown.

Throughout the speech, Poet Dove defines her wishes for the graduating class. Take for example how she wishes, “[the class] Hunger[...,] Hard Work[..., and] Uncertainty.” and then defines them as "continued spiritual and intellectual appetite", "an appreciation for the work that comes before the big show", and "[to] seek [revelation] out" respectively. This would showcase how she wants the graduating class to embrace the unknown because she defines what she believes one needs make the best of their lives. Thus if the graduating class is able to make the best of their lives, they will be able to do good for the world. Although the world is scary, they have already overcame their fear of not making the best of their lives. Since the graduating class of college students had already overcame their fear, they will be able to make the best of what they were given. All of this would ultimately urge the class of 2016 to embrace and conquer the unknown.

In the middle of the speech, Poet Dove exemplifies the work of famous people who follow her definition of Hard Work. Take for example how she exemplifies Nyong'o, "the actress Lupita Nyong’o gives herself homework whenever she has an audition." for her definition of Hard Work. This would showcase how she wants the graduating class to embrace the unknown because by exemplifying her definition she is able to inspire the youthful class. Thus if the graduating class is to be inspired by the success of other artists and humanists, the class will be able to feel the need succeed in the world. Although the class needs to be successful, they have to overcome their fear. Since the graduating class of college students will need to over come fear, they will need to comprehend the unknown. All of this would ultimately urge the class of 2016 to embrace and conquer the unknown.

\subsection{Question 3}

The statement that persuasion via scare tactics is ineffective and causes resentment toward one another is true because people enjoy being treated well and do not enjoy doom and gloom.

People get more with honey than they do with vinegar. During the Nixon Administration the US participated in detente which looked to ease tensions between the US, the USSR and China by seeking positive relationships. President Nixon would become the first president to enter China and later after the success in China the first president since FDR to enter the USSR. The following results were extremely good for countries, in China trade restrictions were lifted and in the USSR the Strategic Arms Limitations Treaty I was signed which limited missile construction and increased sharing of scientific information. As a result the US under the Nixon Administration was able to significantly ease tensions between the communist nations and able to get important work done. Ultimately, persuading with truth and honesty is the best way to get work done.

Doomers and gloomers don’t get listened to because their points revolve around scare tactics. In the 1950s McCarthyism ran rampant. They would attempt to silence and oust all communists. Senator McCarthy led a crusade against the “communist” military, where on live television Americans were able to see the sweat pouring down Senator McCarthy as he lied. Resulting in Senator McCarthy being censured, meaning he was not allowed to speak in the Senate. HUAC even investigated Ronald Reagan for being “associated” with communism. Later Americans would see Reagan become President and effectively end the Cold War. This clearly showcases that the way to get work done is with trust and validity not scare tactics.

\section{2023-1}

\subsection{Question 1}

\subsection{Question 2}

\subsection{Question 3}

There is incredible value in building communities of voice because communities get objectives complete faster and communities help their members.

We saw, we came, we conquered is much better telling of the quote I saw, I came, I conquered. During the Civil War, President Abraham Lincoln fought the Confederates in order to ensure equality for the slaves. In the Civil war he had hundred of thousands of men step up to fight. Some aged 14 such as drummer boys who had to play throughout it all. These true American heroes fought for equality even though it would've benefited them to not. However they did, and we have the land of the free because of the brave. During the Civil Rights Movement of the 1960s, Dr. Martin Luther King Junior lead a community of not just African Americans, but all races, for the March on Washington. There, King would issue his most famous speech, the \textit{I Have a Dream} Speech, where he dreamed for peace and unity. King and his community of got significant Civil Rights legislature passed, providing voting rights, home ownership equality, and financial equality to not just African Americans but to all minorities. They did this together not by themselves.

Acts of selflessness go far especially if you are together on it. In the Vietnam War, soldiers were together attempting to protect democracy and protesters were protesting the Vietnam War together. These soldiers were brave and many of them were 19 years old. These kids were fighting for democracy which brought them together in believing for democracy. While on the other hand, the protesters organized themselves and burnt draft card and engaged in civil disobedience in order to convince the US Government to cut their losses in Vietnam. Whether you were for or against the war in Vietnam there was community built around your opinions which was able to echo it's way to the US President. President Johnson would not campaign for re-election, allowing Nixon to step forward and put an end to Vietnam. It is also important to acknowledge how the Watergate Scandal was also a prime example of how a community of people voiced concerns which lead to Nixon's resignation. These examples show that communities will bring issues forward better than any single person.

\section{2022}

\subsection{Question 1}

\subsection{Question 2}

\subsection{Question 3}

There is value in making timely decisions without knowing 100 percent of the facts because one cannot possibly know 100 percent and bad things will happen one will not act with haste.

No single person knows everything. The countless conflicts in the Middle East are a prime example on why one cannot 100 percent comprehend a conflict. In the six-days war, Israel emerged victorious against a of Middle Eastern countries determined to conquer Israel. In the 1970s the Yom Kippur War between Israel and Egypt, in 2024 the Israel-Palestine conflict. These wars have occurred effectively since the dawn of time. No single person knows what is best for the Middle East. Thus those in charge must use their cabinet to determine the best course of action right now. These timely decisions are important because they ensure the safety of their peoples.

On December 7, 1941, the Japanese unexpectedly and unprovoked attacked the US at Pearl Harbor. This national tragedy could've been avoided by using the partial facts that were known prior to the attack. The Japanese Embassy in Washington, DC had been burning tons of paper documents which is a bad sign. The US had also intercepted messages to said embassy saying how their was no plans to negotiate. This should've lead to the US Government to better integrate information however they didn't learn their lesson. On September 11, 2001, another national tragedy occurred. The twin towers in New York City were targeted by terrorists. The terrorists overtook planes and crashed them into the twin towers. Since then, the US Government has taken steps to secure itself based on time-constraints and all information available at the time. We may never know how many terrorist attacks the US Government has prevented, but they most likely have prevented a lot, that is for sure.

\end{document}
